\documentclass{article}

\usepackage{hyperref}

\title{EPAP FY17}
\author{Jordan Walker}
\date{2016-10-14}

\begin{document}

\maketitle

\section{Summary}

A year in and I am more fully a member of the Data Science team.
This is an exciting time to be a part of this team, as it is more able to be nimble and take advantage of the oportunities that are represented in future priorities for the water mission area.
I'd like to first briefly reflect on my priorities from the last year, followed by a short list of priorities for the next year.
I'm going to attempt to keep my FY17 goals more succinct to prevent the diffusion of focus that results from too many goals without clear priority of which are more important.
Lastly, I will attempt more fully to determine evaluation criteria in order to reflect often on the progress being made and look back during reviews for whether that goal is accomplished.

\section{FY16 Goals}

\subsection{Integration with Data Science team}

The primary difficulty in FY16 was defining the role I would play on the Data Science team and how I could still contribute to work done within the Software Engineering team of which I was a part.
Defining the priorities on projects on which I am to contribute and having better direction on how I am able to contribute would go a long way to reducing the struggle across teams.
I would place this disconnect in cross-team collaboration and communication as the main struggle in fully integrating with the Data Science team.
Progress: 7 out of 10.

\subsection{Collaboration across teams}

This goal goes hand in hand with the above goal, though I think shows less progress.
My previous year goal had attempted to set up a way in which I could be tapped on the shoulder to give aid to a project, either in my computer science capacity on more experimental features, or in my data science capacity to help with a project in a more analytic way.
The reason I suspect for the inability for this goal to be deemed successful is an increasing reliance on process as the driver of progress.
Because there is no clear process set up for inclusion of members of another team, progress was limited to a large extent.
Progress: 4 out of 10.

\subsection{EDGE portfolio development}

Working toward EDGE did not occur to any real extent this year.
The effort that makes the most sense to describe as a marquee product for the porfolio ended up being the VIZLAB development that became more of a priority.
I've put much of this effort on hold and will not list it as a goal for the upcoming year.
Progress: 2 out of 10.

\subsection{Geo Data Portal enhancements}

The goal for better job control to optimize the speed at which jobs can get done, as well as a more scalable system that can lead the way in OWI for a system that scales up and down in response to demand was partially completed this year.
Work across teams was again a major issue in achieving the goal, and the question of future effort on this project put doubts as to whether feature development beyond care and feeding was worth undertaking.
I'm keeping the GDP on my goals list and hope that we can come up with a plan that will keep this project looking to the future.
Progress: 6 out of 10.


\subsection{Visualization framework development}

As a measurable goal, I put forward that I'd like to be involved in another visualization that fully utilizes generalized code for visualization creation.
This goal was partially met by the Great Lakes Microplastics and Climate Fish Habitat visualizations which were used to develop and exercise the platform.
The Hurricane Matthew visualization was the first to fully use the platform, and while there is more progress to be made, the goal of a working platform can be checked off (with minor hesitation).
The goal of a paper was perhaps premature, but I hold that as a future goal.
Progress: 8 out of 10.

\subsection{Hazards algorithm development}

DSASWeb is no longer a project within OWI.
I won't go into too much detail about this goal, other than to say that it would be unfair to say anything other than that it was not met.
Progress: 1 out of 10.

\section{FY17 Goals}

\subsection{VIZLAB platform}

Now that the VIZLAB package is somewhat stable, it will be a major priority of mine to polish the edges and transfer knowledge so that it is an quick and easy to use by any developer tasked with working on visualizations.
There will need to be additional features to get beyond where we are, especially related to figuring out the proper infrastructure, so some major effort on this will need to make its way into the work plan of the team.

Part of this includes getting it ready for the possibility of an internship team being tasked with creating a visualization as a summer project.
This will be a major test of the platform, and I will need to be available to assist wherever needed as part of this being a success.
As a side goal within this effort would be determining some expertises that are not covered by the current Data Science team, but might be useful particularly within the Visualization theme.

Additionally the Data Science team will continue to push ahead with new and compelling visualizations.
I'd like to make a rough goal of one visualization per quarter at this point until we determine that the frequency should be more or less frequent.
Another metric I want to keep on top of is the work hours and clock hours required from initialization to publication of a visualization.
Lowering this number by relying on a foundational platform which removes the major burdens of development is a continued goal that I'd like to make progress on.

\subsection{Geo Data Portal}

This goal is the only other one I carried over from last year.
I feel that the approach last year was attempting to bridge the gap between Software Engineering and Data Science, but the attempt did not work to that end.
This year I'd like to lay out the goal in more of a manner that I have more control over, which is to say the development that I will do myself.
As a general shift, I'd also like to move the GDP work from maintaining the status quo with some improvements to a more comprehensive look into what is being used for and how that can be improved, as well as adding value by exposing it to more users and use cases.

The first effort of importance is taking a bit more of a planning role on the technical work and requirements.
Features should be planned based on what is valuable to current and future users.
For the current users we have data that can be used to figure out which features are of value and what might be current hang ups with the  system.
For the future users, we can use strategic goals to determine what fields the GDP should be playing in and guide feature work in that direction.


\subsection{Enhance outreach and training ability}

\subsection{HTC and HPC proficiency}

\subsection{Solidify shared infrastructure presence for Data Science}

\end{document}
