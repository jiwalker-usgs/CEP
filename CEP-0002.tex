CEP: 0002
Title: On a Cross-functional Data Science Team
Version: $Revision$
Last-Modified: $Date$
Author: Jordan Walker <jiwalker@usgs.gov>
Status: Active
Type: Process
Content-Type: text/x-rst
Created: 17-Feb-2015
Post-History: 


Abstract
========

Diversity and communication are key characteristics of effective interdisciplinary teams [_1].  CIDA teams are generally aligned according to disciplinary and skill set backgrounds (developers, system administrators, and project leads), despite operating in an interdisciplinary field, in which software engineers, scientists, project managers, and system administrators work together to produce technology solutions.  Historically, communication barriers between CIDA teams have prevented beneficial overlap of information and ideas.  In the interest of maintaining and enhancing a sustainable and effective CIDA, a formal mechanism to increase communication channels between CIDA teams may be necessary. 

I propose adding a member of the developer team to the newly formed data
science team in order to increase communication between these groups.  This position will act to prevent this group developing as a ‘silo’, which could negatively impact CIDA overall.  Such a position will benefit the data science team by directing computer science expertise to the goals of the team.  
Finally, alternative paths for computer scientist career advancement may be exposed by this change; demonstrating these alternative paths will help for recruitment and retention of computer scientists in the future.

Rationale
=========

Assembling effective teams to tackle complex challenges requires mixing div-
erse backgrounds and a blend of specialists and generalists [_1]. The goal 
of team formation within CIDA should not be to collect a group of people
with the same background, but rather to collect a group of people with mutual 
respect and complementary skills, that are willing to converge on a shared 
approach to solving the problems that are presented to CIDA.  The existing 
data science team is utilizing some of CIDA's best-practices combined with 
novel scientific research to establish a new approach to the challenges CIDA 
is attempting to counter.  With this approach, the team has an opportunity 
to distinguish itself within USGS, but will best accomplish its goals with 
a cross-functional team.

One of the directives of the data science team is to publish research that
highlights the work of CIDA, the OWI, and the USGS.  Much of this work will
be done by utilizing APIs produced within and outside of the USGS.  A developer with firsthand experience with APIs can contribute to data science research by most effectively leveraging these technologies, troubleshooting issues, and by bringing to the table knowledge of APIs that might otherwise be unknown to the data science group. The data science team, and CIDA as a whole, will also benefit by the input of a developer on architectural decisions made by the data science team, and by integration with existing CIDA work. 

The goal of ‘cross-functional’ teams is prevalent within CIDA, but defining
that goal is somewhat difficult.  One definition that seems to be making
its rounds is that everyone on a team is able to take on any task that is
presented to the team.  While this is a goal that has its merits, a
more widely accepted definition of a cross-functional team is a "group of 
people with different functional expertise working toward a common goal" 
[_2].  By being a cross-functional team, the data science team will 
integrate better with CIDA and be more able to accomplish its goals.

A growing issue within the development group is recruitment.  Qualified
persons are difficult to find, and harder to hire.  One point that might
aid in recruitment would be for different career paths that one could
take.  The RGE and EDGE [_3] programs allow for advancement based on
accomplishments in research and development.  Currently this is an untested
approach for computer scientists, especially for CIDA.  Having this option
available would help especially if looking for candidates with a more
academic background (e.g. Computer Science or Electrical Engineering graduate
students).  Currently this path would most likely be set through efforts
taken on by the data science team.

Duties
======

As a member of the data science team, the developer role would include:

- Act as a liaison between the data science team and the other developers.

- Keep track of services and APIs that might be of use to data science
team and advocate for any additions that might be helpful.

- Act as a developer on projects that require the skill-set of a
traditional developer but fit within the mission of the data science team.

- Keep up with trends in development of methods to work with big data,
complex models, and scalable processing.

- Provide architectural reviews for software developed within the data
science team.

- Perform research related to work being done within CIDA, OWI, and the
Water mission area.

- Contribute to an RGE/EDGE portfolio and be evaluated for a position.

Transition
==========

Acknowledging that this is a bit of a change, it would be best to plan
the transition to avoid adversely affecting ongoing projects.  The developer
switching should start out at around 25% on the data science team, joining
in on meetings while still working on projects already committed to.  Then
as planning and funding allows the allocation can increase to an eventual
100%.

Additionally, the increased capacity on the data science team will allow
for projects that fit with the goals of the team to be added and tackled
as a joint effort between functional groups.  The flexibility this will
offer should ensure a self-sufficient data science team into the future.

Conclusion
==========

The establishment of a data science team was a smart move for CIDA and the
OWI.  Furthering its effectiveness through tighter integration with the
rest of CIDA will benefit both the team and CIDA.  Working within a different
work structure allows for multiple approaches to solving the science problems
presented to the USGS.  Introducing multiple functions into the separate
approaches allows for fully functional teams which can complete tasks
independent of others, while preserving the channels of communication
that make CIDA a unique team and place to work.

References
==========

.. [1] Cheruvelil, K. S., P. A. Soranno, K. C. Weathers, P. C. Hanson, S. J. 
Goring, C. T. Filstrup, and E.K. Read (2014), Creating and maintaining 
high-performing collaborative research teams: the importance of diversity 
and interpersonal skills, Front. Ecol. Environ., 12(1), 31–38, 802.

.. [2] Cross-functional team, Wikipedia.
	(http://en.wikipedia.org/wiki/Cross-functional_team)

.. [3] USGS Research and Development Guidance.
	(http://www.usgs.gov/humancapital/hr/rgeg.html)
Copyright
=========

This document has been placed in the public domain.



..
   Local Variables:
   mode: indented-text
   indent-tabs-mode: nil
   sentence-end-double-space: t
   fill-column: 70
   coding: utf-8
   End:
