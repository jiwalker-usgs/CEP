CEP: XXX
Title: A CIDA Docker Proof of Concept
Version: $Revision$
Last-Modified: $Date$
Author: Jordan Walker <jiwalker@usgs.gov>
Status: Active
Type: Process
Content-Type: text/x-rst
Created: 06-Jan-2015
Post-History: 


Abstract
========

There are many options out there in the worlds of containerization and
configuration management of services.  As a group, CIDA [1]_ has several
types of projects that can benefit from the different configuration
management solutions that are out there.  In particular applications that
require particular package dependencies in order to operate correctly can
make use of a configuration management solution to ensure that those
dependencies are available to the application.

Docker [2]_ is one such option available, and has been growing in popularity
of late.  I propose a proof of concept Docker setup on CIDA infrastructure
in order to determine whether it should be considered for general
applicability across many applications as a containerization and
configuration management mechanism.

The application chosen for this stage of our use of docker will be a simple
service that is non-stateful and particularly fits the Docker use-case well.
The result of this proof of concept will be a better understanding of the
security implications of using Docker in our infrastructure and some
guidance on what applications and services fit the Docker model of
containerization and configuration management.

Rationale
=========

While other configuration management approaches are being used and pursued
within our group [3]_, [4]_, [5]_ Docker provides a unique approach to the
problem that benefits certain applications in ways that we do not have
answers to in any current system.  A proof of concept will go a long way
to show the benefits and challenges of Docker or similar systems.

The following features of Docker are particularly of interest at this point:

- Running multiple sandbox-ed containers within a single Virtual Machine

- Having an artifact that represents all dependencies of a service

- Inheritance of configuration so that many containers can be based on a
  single base setup

- Ability to have developer determined commands run for setup that would
  otherwise be performed manually by admin staff

Some of these features are available in other approaches that have been
suggested, but Docker has simple mechanisms for all these points.  If 
there is was a best-practice for this type of task at CIDA it would not
be needed to come up with another solution, but to date nothing is clear
in its use for this particular need.

Service Containerization
========================

The goal in making containers for services is to have everything the service
needs to run be pre-baked into a container.  Then the container is deployed
and able to run the service with minimal adjustments.  Such a container can
be achieved through virtualization, but recent changes have made Linux
Containers [6]_ an appealing alternative.  Docker fits into the latter group
and offers lower overhead when compared to virtual machines.

When using containers to avoid dependencies conflicts in system dependencies,
system level containers are required.  Alternate approaches to sandbox-ed
environments are worth discussing for the benefit they bring, but none solve
the system dependency problem.  The Java classloader allows for different
services to use different versions of libraries, python virtualenv [7]_
installs isolated python environments to achieve the same end, and packrat [8]_
is an R package that offers similar functionality.

Configuration Management
========================

Configuration of a service involves configuration files as well as
installation of all the dependencies.  This can be accomplished through a
combination of virtualization and automation scripts or by putting all the
configuration in the artifact itself.  The latter approach is that taken by
Docker and fits well into a continuous integration setup where the result of
a build is an artifact that can work its way to production.

A point should be made that Docker isn't configuration management in the sense
of other configuration management tools, but rather by specifying a build file
for what should be included in the container.  For some projects this will not
be sufficient, but for our proof of concept use-case with a stateless service
this will do just fine.

Lastly, when applying configuration to servers, often times root access is
necessary.  The distinction of what is run with root privileges should clearly
be made and reviewed in conjunction with the security team to avoid possible
breaches.  Any solution we come up with will likely have this feature, and
Docker in particular provides a good mechanism for managing this.

Requirements
============

Having made the point that this is a worthwhile exercise, here are the
requirements from the author for such a solution.

- A single machine with Docker host installed

- A registry (remote or local to host) that images can be pushed to or pulled
  from

- Ability for devs (or eventually Jenkins) to run ``docker`` commands

  * Some commands may be restricted for security reasons, in particular

  * docker run should not allow volumes to be created as there is no easy
    mechanism for preventing root volumes from being mounted

  * docker run must allow at least flags -d -P for devs to get use from this

- A process for approving Dockerfiles for building

- A means for approved images to be uploaded to registry

- Ports in the range 49153 and 65535 should be opened for other services to
  be able to communicate with the docker container services

- The following guidelines will be followed by developers of Dockerfiles

  * The de-escalation from root shall occur as soon as possible

  * A non-root user shall run all services inside the container

  * Best practices [9]_ shall be followed

    + Exception: recommendations to use official repositories should be
      ignored because we are going to have our own registry of trusted images

Conclusion
==========

A proof of concept Docker setup will give valuable insight into how this tool
might fit into CIDA's infrastructure.  Benefits such as lower overhead and
improved consistency of services promise to improve our ability to deliver
quality software.  Lessons learned from this initial exploration will inform
us as we look to future work in developing microservices.

Generally it is wise to avoid the "next big thing" in technology as it often
leads to fragmented stacks and products that have not withstood the test of
time.  Docker is being touted as the "next big thing", so we should tread
softly, but the promise it brings for many areas of our development make it
too hard to avoid it much longer.

References
==========

.. [1] Center for Integrated Data Analytics
   (http://cida.usgs.gov/)

.. [2] Docker, Docker Introduction, Docker, Inc.
   (http://www.docker.com/whatisdocker/)

.. [3] CFEngine, CFEngine AS
   (http://cfengine.com/)

.. [4] Ansible, Simple IT Automation, Ansible, Inc.
   (http://www.ansible.com/home)

.. [5] Chef, IT automation for speed and awesomeness, Chef Software, Inc,
   (https://www.chef.io/chef/)

.. [6] Linux Containers
   (https://linuxcontainers.org/)

.. [7] Virtualenv
   (https://virtualenv.pypa.io/en/latest/)

.. [8] packrat
   (https://github.com/rstudio/packrat)

.. [9] Dockerfile Best Practices, Docker, Inc.
   (https://docs.docker.com/articles/dockerfile_best-practices/)

Copyright
=========

This document has been placed in the public domain.



..
   Local Variables:
   mode: indented-text
   indent-tabs-mode: nil
   sentence-end-double-space: t
   fill-column: 70
   coding: utf-8
   End:
